\documentclass[10pt,a4paper]{article}
\usepackage[utf8]{inputenc}
\usepackage{amsmath}
\usepackage{amsfonts}
\usepackage{amssymb}

\title{Parallel Computing Exercise 1}
\date{}

\begin{document}

\maketitle

\section{Task 1 - Run-Time influencing factors}
\begin{itemize}
\item Algorithmic complexity of the program (notation: $O(g(n))$, or $\Theta(g(n))$)\\
This is a very high factor for program runtime as this is an estimation of the 'number of steps' that the algorithm has to make in order to produce the output. The importance of algorithmic complexity can be seen by the fact that one of the famous 'Millenium Problems' is indeed the question if $P=NP$.
\item Input size for the algorithm (the $n$ in $O(g(n))$)\\

\end{itemize}

\end{document}